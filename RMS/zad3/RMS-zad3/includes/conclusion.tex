Úlohou tohto zadania bolo namodelovať a simulovať riadenie dvojmotorového systému s pružným členom.

Na riadenie sme použili PI regulátor, ktorého parametre sme hľadali pomocou genetického algoritmu. Súčet odchýlky regulovanej sústavy a akčnej veličiny sme použili ako fitnes funkciu ktorú sme minimalizovali. Výsledkom algoritmu boli parametre regulátora. \ref{GA}. Následne sme s nájdenými parametrami otestovali rôzne prevody \ref{prevody}, koeficienty tuhosti \ref{tuhost} a koeficienty tlmenia \ref{tlmenie}.

V sekcii prevody \ref{prevody} môžeme pozorovať, že regulátor si vedel dobre poradiť aj s inými prevodovkami na ktoré bol navrhnutý. Systém mal zo začiatku oneskorenie čo mohlo spôsobovať suché trenie a pomalší nábeh systému na žiadanú veličinu a keď že hodnoty uhlovej rýchlosti na úseku 1 až 1.5 sekundy neboli vysoké tak genetický algoritmus zmenu nebral ako veľkú chybu a zároveň sme fitnes funkciou chceli obmedziť vplyv akčného zásahu takže akčný zásah mal rovnakú váhu ako regulačná odchýlka, čo taktiež spôsobilo oneskorenie nábehu systému.

Testovali sme aj vplyv koeficientu tuhosti \ref{tuhost} pri prevodovom stupni 20. Menili sme ho s faktorom 10, čiže z 0.5 na 0.05 atď. a na závislosti rýchlosti záťaže od času môžme pozorovať, že systém reagoval na zmenu tuhosti veľmi dobre. Vyššou rýchlosťou sa musel točiť hnací motor aby pri nízkej hodnote $K_e$, hnaný motor dosiahol žiadanú rýchlosť. Z čoho vyplýva, že znižovaním koeficientu tuhosti je systém viac pružný, čo znamená aby sa hnaný motor rozbehol, hnací motor dosiahne vyššiu rýchlosť.

Zmenou koeficientu tlmenia \ref{tlmenie}, sme zistili, že mal vplyv na rozbeh hnaného motora. Menili sme ho s faktorom 10, čiže z 0.5 na 0.05 atď. a ak bol väčší ako koeficient pre ktorý bol regulátor navrhnutý tak pri rozbehu systému sme nemali kmitanie a ak sme koeficient zväčšovali rýchlosť hnaného motora viac oscilovala pri rozbehu.

Momenty motora vo všetkých experimentoch sa nedostali na maximum, čím sme obmedzili saturáciu. Pre lepšie dosiahnuté výsledky aby motor mal väčší moment by som odporučil v genetickom algoritme zmenšiť váhu akčnej veličiny.